% Options for packages loaded elsewhere
% Options for packages loaded elsewhere
\PassOptionsToPackage{unicode}{hyperref}
\PassOptionsToPackage{hyphens}{url}
\PassOptionsToPackage{dvipsnames,svgnames,x11names}{xcolor}
%
\documentclass[
  authoryear,
  review,
  1p]{elsarticle}
\usepackage{xcolor}
\usepackage{amsmath,amssymb}
\setcounter{secnumdepth}{5}
\usepackage{iftex}
\ifPDFTeX
  \usepackage[T1]{fontenc}
  \usepackage[utf8]{inputenc}
  \usepackage{textcomp} % provide euro and other symbols
\else % if luatex or xetex
  \usepackage{unicode-math} % this also loads fontspec
  \defaultfontfeatures{Scale=MatchLowercase}
  \defaultfontfeatures[\rmfamily]{Ligatures=TeX,Scale=1}
\fi
\usepackage{lmodern}
\ifPDFTeX\else
  % xetex/luatex font selection
\fi
% Use upquote if available, for straight quotes in verbatim environments
\IfFileExists{upquote.sty}{\usepackage{upquote}}{}
\IfFileExists{microtype.sty}{% use microtype if available
  \usepackage[]{microtype}
  \UseMicrotypeSet[protrusion]{basicmath} % disable protrusion for tt fonts
}{}
\makeatletter
\@ifundefined{KOMAClassName}{% if non-KOMA class
  \IfFileExists{parskip.sty}{%
    \usepackage{parskip}
  }{% else
    \setlength{\parindent}{0pt}
    \setlength{\parskip}{6pt plus 2pt minus 1pt}}
}{% if KOMA class
  \KOMAoptions{parskip=half}}
\makeatother
% Make \paragraph and \subparagraph free-standing
\makeatletter
\ifx\paragraph\undefined\else
  \let\oldparagraph\paragraph
  \renewcommand{\paragraph}{
    \@ifstar
      \xxxParagraphStar
      \xxxParagraphNoStar
  }
  \newcommand{\xxxParagraphStar}[1]{\oldparagraph*{#1}\mbox{}}
  \newcommand{\xxxParagraphNoStar}[1]{\oldparagraph{#1}\mbox{}}
\fi
\ifx\subparagraph\undefined\else
  \let\oldsubparagraph\subparagraph
  \renewcommand{\subparagraph}{
    \@ifstar
      \xxxSubParagraphStar
      \xxxSubParagraphNoStar
  }
  \newcommand{\xxxSubParagraphStar}[1]{\oldsubparagraph*{#1}\mbox{}}
  \newcommand{\xxxSubParagraphNoStar}[1]{\oldsubparagraph{#1}\mbox{}}
\fi
\makeatother


\usepackage{longtable,booktabs,array}
\newcounter{none} % for unnumbered tables
\usepackage{calc} % for calculating minipage widths
% Correct order of tables after \paragraph or \subparagraph
\usepackage{etoolbox}
\makeatletter
\patchcmd\longtable{\par}{\if@noskipsec\mbox{}\fi\par}{}{}
\makeatother
% Allow footnotes in longtable head/foot
\IfFileExists{footnotehyper.sty}{\usepackage{footnotehyper}}{\usepackage{footnote}}
\makesavenoteenv{longtable}
\usepackage{graphicx}
\makeatletter
\newsavebox\pandoc@box
\newcommand*\pandocbounded[1]{% scales image to fit in text height/width
  \sbox\pandoc@box{#1}%
  \Gscale@div\@tempa{\textheight}{\dimexpr\ht\pandoc@box+\dp\pandoc@box\relax}%
  \Gscale@div\@tempb{\linewidth}{\wd\pandoc@box}%
  \ifdim\@tempb\p@<\@tempa\p@\let\@tempa\@tempb\fi% select the smaller of both
  \ifdim\@tempa\p@<\p@\scalebox{\@tempa}{\usebox\pandoc@box}%
  \else\usebox{\pandoc@box}%
  \fi%
}
% Set default figure placement to htbp
\def\fps@figure{htbp}
\makeatother





\setlength{\emergencystretch}{3em} % prevent overfull lines

\providecommand{\tightlist}{%
  \setlength{\itemsep}{0pt}\setlength{\parskip}{0pt}}



 
\usepackage[]{natbib}
\bibliographystyle{elsarticle-harv}


\makeatletter
\@ifpackageloaded{caption}{}{\usepackage{caption}}
\AtBeginDocument{%
\ifdefined\contentsname
  \renewcommand*\contentsname{Table of contents}
\else
  \newcommand\contentsname{Table of contents}
\fi
\ifdefined\listfigurename
  \renewcommand*\listfigurename{List of Figures}
\else
  \newcommand\listfigurename{List of Figures}
\fi
\ifdefined\listtablename
  \renewcommand*\listtablename{List of Tables}
\else
  \newcommand\listtablename{List of Tables}
\fi
\ifdefined\figurename
  \renewcommand*\figurename{Figure}
\else
  \newcommand\figurename{Figure}
\fi
\ifdefined\tablename
  \renewcommand*\tablename{Table}
\else
  \newcommand\tablename{Table}
\fi
}
\@ifpackageloaded{float}{}{\usepackage{float}}
\floatstyle{ruled}
\@ifundefined{c@chapter}{\newfloat{codelisting}{h}{lop}}{\newfloat{codelisting}{h}{lop}[chapter]}
\floatname{codelisting}{Listing}
\newcommand*\listoflistings{\listof{codelisting}{List of Listings}}
\makeatother
\makeatletter
\makeatother
\makeatletter
\@ifpackageloaded{caption}{}{\usepackage{caption}}
\@ifpackageloaded{subcaption}{}{\usepackage{subcaption}}
\makeatother
\journal{Psychological methods}
\usepackage{bookmark}
\IfFileExists{xurl.sty}{\usepackage{xurl}}{} % add URL line breaks if available
\urlstyle{same}
\hypersetup{
  pdftitle={Finding Thurstone: modeling comparative judgment data with R (and Stan)},
  pdfauthor={Jose Manuel Rivera Espejo; Tine van Daal; Sven De Maeyer; Steven Gillis},
  pdfkeywords={tutorial, causal inference, bayesian
inference, thurstonian model, comparative judgement, statistical
modeling},
  colorlinks=true,
  linkcolor={blue},
  filecolor={Maroon},
  citecolor={Blue},
  urlcolor={Blue},
  pdfcreator={LaTeX via pandoc}}


\setlength{\parindent}{6pt}
\begin{document}

\begin{frontmatter}
\title{Finding Thurstone: modeling comparative judgment data with
\texttt{R} (and \texttt{Stan})}
\author[1]{Jose Manuel Rivera Espejo%
\corref{cor1}%
}
 \ead{JoseManuel.RiveraEspejo@uantwerpen.be} 
\author[1]{Tine Daal%
%
}
 \ead{tine.vandaal@uantwerpen.be} 
\author[1]{Sven Maeyer%
%
}
 \ead{sven.demaeyer@uantwerpen.be} 
\author[2]{Steven Gillis%
%
}
 \ead{steven.gillis@uantwerpen.be} 

\affiliation[1]{organization={University of Antwerp, Training and
education sciences},,postcodesep={}}
\affiliation[2]{organization={University of
Antwerp, Linguistics},,postcodesep={}}

\cortext[cor1]{Corresponding author}




        
\begin{abstract}
The classical BTL analysis has become the standard approach for
analyzing comparative judgment (CJ) data because it provides a simple
method for measuring traits and conducting related analysis. This
simplicity arises from two key features. First, the approach relies on
the Bradley-Terry-Luce (BTL) model to estimate latent traits. Second, it
uses of ad hoc procedures to conduct related analysis. However, recent
studies question whether the BTL model assumptions hold in contemporary
CJ applications and whether the ad hoc procedures effectively fulfill
their intended analytical goals.

To address these concerns, Rivera and colleagues
\citeyearpar{Rivera_et_al_2025} proposed an approach that extends the
general form of Thurstone's law of comparative judgment. The approach
enables the development of a model tailored to the assumed
data-generating process of the CJ system under study, eliminating the
need for simplifying assumptions. Moreover, by integrating measurement
and inference within a single analytical framework, it also removes the
dependence on ad hoc analytical procedures. Despite these advantages,
the approach still requires empirical validation.

Thus, this study empirically validates the proposed
Information-Theoretical model for CJ, benchmarked against the classical
BTL analysis, and demonstrates its practical implementation. The
document includes a structured tutorial based on a simulated
speech-quality dataset, providing guidance on data simulation, prior
specification, model estimation, and interpretation using \texttt{Stan},
\texttt{R}, and the interface packages \texttt{cmdstan} and
\texttt{brms}. Ultimately, the study equips researchers with practical
tools to apply the model to more complex CJ studies.
\end{abstract}





\begin{keyword}
    tutorial \sep causal inference \sep bayesian
inference \sep thurstonian model \sep comparative judgement \sep 
    statistical modeling
\end{keyword}
\end{frontmatter}
    

\section{Introduction}\label{sec-introduction}

\emph{Comparative judgment} (CJ) has emerged as a valuable methodology
for measuring latent traits across diverse fields, including education
\citep{Kimbell_2012, Jones_et_al_2015, vanDaal_et_al_2016, Bartholomew_et_al_2018},
political sciences \citep{Zucco_et_al_2019}, linguistics
\citep{Boonen_et_al_2020}, and criminology \citep{Seymour_et_al_2025}.
In CJ studies, judges actively compare pairs of stimuli to determine
which stimulus exhibits more of the latent trait of interest
\citep{Thurstone_1927a, Thurstone_1927b}.

A specific data analysis workflow has become the standard approach for
analyzing CJ data \citep[see, e.g.,][]{Thwaites_et_al_2024}. In this
study, we refer to this workflow as the classical BTL analysis or
\emph{CBTL analysis}. Researchers favor this approach because it
provides a simple method for measuring traits and conducting related
analyses \citep{Andrich_1978, Pollitt_2012b}. This simplicity arises
from two key features. First, the approach relies on the
Bradley-Terry-Luce (BTL) model \citep{Bradley_et_al_1952, Luce_1959} to
estimate latent traits. The model facilitates trait estimation by
imposing several simplifying assumptions about traits, judges, and
stimuli present in CJ assessments \citep{Thurstone_1927b, Bramley_2008}.
Second, the approach uses ad hoc procedures to conduct related analysis,
including data summarization and statistical inference
\citep{Pollitt_2012b}.

Recent studies, however, question whether the assumptions of the BTL
model hold in contemporary CJ applications and whether the ad hoc
procedures achieve their intended analytical goals
\citep{Bramley_2008, Kelly_et_al_2022, Rivera_et_al_2025}. For instance,
Rivera and colleagues \citeyearpar{Rivera_et_al_2025} argue that while
the assumptions of equal dispersions and zero correlations between
stimuli simplify trait measurement, they may fail to represent complex
traits or heterogeneous stimuli adequately
\citep{Thurstone_1927a, Andrich_1978, vanDaal_et_al_2016, Lesterhuis_et_al_2018, Chambers_et_al_2022}.
As a result, such assumptions can compromise the reliability and
accuracy of trait estimates
\citep{Ackerman_1989, Zimmerman_1994, McElreath_2020, Wu_et_al_2022, Miller_2023, Hoyle_et_al_2023}.
Moreover, the same authors note that although ad hoc procedures simplify
data analyses, the use of untested methods can also undermine the
validity of inferences derived from CJ data
\citep{McElreath_2020, Kline_et_al_2023, Hoyle_et_al_2023}.

To address these concerns, \citet{Rivera_et_al_2025} proposed an
approach that extends the general form of Thurstone's law of comparative
judgment \citep{Thurstone_1927a, Thurstone_1927b}, referred to as the
Information-Theoretical model for CJ (hereafter, ITCJ analysis). This
approach leverages causal and Bayesian inference methods to combine
Thurstone's core theoretical principles with key design features of CJ
assessment. By doing so, it enables the development of a model tailored
to the assumed data-generating process of the CJ system under study.
This tailoring effectively removes the need to rely on the simplifying
assumptions of the BTL model. Moreover, by integrating measurement and
inference within a single analytical framework, the approach also
eliminates the dependence on ad hoc analytical procedures.

\subsection{Research goals}\label{sec-introduction_goals}

The ITCJ analysis \citep{Rivera_et_al_2025} shows theoretical promise
for yielding reliable trait estimates and accurate statistical
inferences. However, as noted by the authors, this promise has not yet
been empirically validated. Thus, the present study addresses this gap
by pursuing two closely related research goals. The first goal is to
\emph{empirically validate} the proposed ITCJ analysis by evaluating the
accuracy and reliability of its trait estimates and inference
parameters, benchmarked against the CBTL analysis.

The second goal emerges as a practical byproduct of this validation:
\emph{to demonstrate how to implement the model in practice}. To this
end, the document provides a structured tutorial based on a simulated
speech-quality dataset, offering guidance on data simulation, prior
specification, model estimation, and interpretation using
\texttt{Stan}\citep{Stan_2026a, Stan_2026b}, \texttt{R} \citep{R_2015},
and the interface packages \texttt{cmdstan} \citep{Gabry_et_al_2025b}
and \texttt{brms} \citep{Burkner_2017, Burkner_2018}. By combining model
validation and practical instruction, the study evaluates the
methodological performance of the ITCJ analysis and equips researchers
with practical tools to apply it to more complex CJ studies.

The remainder of this manuscript is organized into five sections.
Section~\ref{sec-theory} reviews the two analytical approaches commonly
applied to CJ data: the CBTL and ITCJ analyses.
Section~\ref{sec-methods} details the assumed data-generating process
for the simulated dataset, the simulation procedure, the practical
implementation of each analytical approach, and the evaluation criteria
aligned with the research goals. Section~\ref{sec-results} presents the
data description and modeling results. Section~\ref{sec-discussion}
interprets the findings, outlines future research directions, and
considers the study limitations. Finally, Section~\ref{sec-conclusion}
offers the concluding remarks.

\section{A tale of two analytical approaches}\label{sec-theory}

Pairwise comparison data, and more specifically CJ data, can be analyzed
using two main approaches: the CBTL and the ITCJ analysis. The CBTL
approach applies a sequence of separate analytical steps to estimate
traits and draw inferences. In contrast, the ITCJ analysis uses a
single, systematic, and integrated approach to achieve the same
objectives. This section describes the two approaches in detail.

\subsection{The CBTL analysis}\label{sec-theory_CBTL}

The CBTL approach implements a sequence of separate analytical steps,
each serving a distinct purpose
\citep{Pollitt_2012a, Pollitt_2012b, Jones_et_al_2019, Boonen_et_al_2020, Chambers_et_al_2022, Bouwer_et_al_2023}.
This multi-step procedure typically unfolds as follows. First, analysts
apply the BTL model to the CJ data to produce two outputs: (1) point
estimates of stimulus traits along with their standard errors, and (2)
residuals at the stimulus level. These outputs provide the foundation
for the subsequent analyses.

Second, researchers summarize or fit regression models to the stimulus
point estimates. This step serves multiple purposes, including
aggregating stimulus-level estimates to the individual level,
partitioning variability between and within individuals, and drawing
inferences about factors that influence trait values. For example,
\citet{Boonen_et_al_2020} applied a multilevel regression model to the
stimulus point estimates to examine whether children's age or hearing
status affects their intelligibility scores.

Third, analysts summarize or fit regression models to the BTL residuals.
This step helps to aggregate the remaining variability at the judge
level, partition residual variability between and within judges, test
for systematic biases, and identify potential misfitting judgments,
stimuli, or judges. For instance, \citet{Wu_2025} fitted an analysis of
variance (ANOVA) model to the infit statistic for each rater to examine
the potential effects of raters' expertise on their judgments. The infit
statistics is a weighted average of the squared Pearson residuals
\citep{Wright_et_al_1982}.

While this stepwise approach is the standard practice in fields such as
education \citep[see,][]{Wu_2025} and linguistics
\citep[see,][]{Thwaites_et_al_2024}, it presents several limitations.
First, each stage treats outputs from previous steps as fixed data,
rather than acknowledging their status as uncertain parameter estimates.
Failure to account for this uncertainty can introduce bias and decrease
the precision of inferences. The direction and magnitude of these biases
can be unpredictable: results may be attenuated, amplified, or remain
unaffected depending on the uncertainty in the scores and the actual
effects being tested
\citep{McElreath_2020, Kline_et_al_2023, Hoyle_et_al_2023}. Moreover,
the loss of precision diminishes statistical power and increases the
likelihood of committing type I or type II errors
\citep{McElreath_2020}. Second, the procedure lacks theoretical
coherence, as it combines models with different underlying assumptions.
For example, third-step analyses treat residuals as outputs that
ostensibly capture deviations from expected comparison outcomes,
potentially reflecting judge-specific tendencies or idiosyncrasies in
judgments, without providing evidence on whether these assumptions hold.

\subsection{The ITCJ analysis}\label{sec-theory_ITCJ}

\newcommand{\dsep}{\:\bot\:}
\newcommand{\ndsep}{\:\not\bot\:}

The ITCJ analysis addresses the aforementioned limitations by providing
a unified and systematic approach to analyzing CJ data
\citep{Rivera_et_al_2025}. It starts with a general \emph{Directed
Acyclic Graph (DAG)} and a corresponding \emph{Structural Causal Model
(SCM)} \citep{Morgan_et_al_2014, Gross_et_al_2018, Neal_2020}, which
together establish a coherent theoretical foundation for CJ analysis by
explicitly representing the relationships among observed judgments,
discriminal differences, stimulus traits, individual traits, judge
biases, and the sampling and comparison mechanisms. Next, the approach
adapts the general SCM and DAG to the assumed data-generating process of
the CJ system under study. Then, it derives one or more bespoke
\emph{probabilistic} and \emph{statistical} models tailored to that
system. Finally, it uses one or more statistical models to estimate
traits and conduct statistical inference. Figure~\ref{fig-ITCJ_dag}
illustrates an example of a general DAG structure.

\begin{figure}[H]

\centering{

\includegraphics[width=0.72\linewidth,height=\textheight,keepaspectratio]{3_results/theory/figures/png/CJ_TM_15.png}

}

\caption{\label{fig-ITCJ_dag}ITCJ analysis: DAG}

\end{figure}%

In this representation, \(O_{R}\) denotes the observed judgment outcome
vector, and \(D_{R}\) represents the discriminal difference vector.
\(T_{IA}\) captures the vector of stimulus traits, and \(T_{I}\)
represents the vector of individual traits. The vector of judgment
biases is represented by \(B_{JK}\), while the vector of judges' biases
by \(B_J\). Covariates at the stimulus and individual levels appear as
\(X_{IA}\) and \(X_{I}\), respectively; while \(Z_{JK}\) and \(Z_{J}\)
represent covariates at the judgment and judge levels. The error terms
(\(e_{IA}\), \(e_{I}\), \(e_{JK}\), \(e_{J}\)) capture residual
variation at each level.

The DAG also represents the sampling and comparison mechanisms as the
vectors \(S\) and \(C\), two conditioned variables that determine how
population outcomes become ``observed'' outcomes. Importantly, the DAG
depicts \(S\) and \(C\) as independent from all other variables by
showing no arrows pointing into them. Regarding \(S\), this indicates
that the DAG applies to \emph{Simple Random Sampling (SRSg)} designs,
where each repeated judgment, judge, stimulus, and individual has an
equal probability of inclusion within their respective groups
\citep{Lawson_2015}. Regarding \(C\), the DAG applies to \emph{Random
Allocation Comparative Designs} \citep{Bramley_2015} or \emph{Incomplete
Block Designs} \citep{Lawson_2015}, where every repeated judgment has an
equal chance of being included in the sample.

Researchers can then translate this DAG representation into SCMs and
probabilistic forms that express the joint distribution of a complex CJ
system as a product of simpler \emph{conditional probability
distributions (CPDs)}
\citetext{\citealp{Pearl_et_al_2016}; \citealp{Neal_2020}; \citealp[see
also][section 5]{Rivera_et_al_2025}}, as illustrated in
Figure~\ref{fig-ITCJ}.

\begin{figure}

\begin{minipage}{0.50\linewidth}

\centering{

\[
\begin{aligned}
  O_{R} & := f_{O}(D_{R}, S, C) \\ 
  D_{R} & := f_{D}(T_{IA}, B_{JK}) \\
  T_{IA} & := f_{T}(T_{I}, X_{IA}, e_{IA}) \\
  T_{I} & := f_{T}(X_{I}, e_{I}) \\
  B_{JK} & := f_{B}(B_{J}, Z_{JK}, e_{JK}) \\
  B_{J} & := f_{B}(Z_{J}, e_{J}) \\ \\
  e_{I} & \:\bot\:\{ e_{J}, e_{IA}, e_{JK} \} \\
  e_{J} & \:\bot\:\{ e_{IA}, e_{JK} \} \\
  e_{IA} & \:\bot\:e_{JK}
\end{aligned}
\]

}

\subcaption{\label{fig-ITCJ_scm}SCM}

\end{minipage}%
%
\begin{minipage}{0.50\linewidth}

\centering{

\[
\begin{aligned}
  & P( O_{R} \mid D_{R}, S, C ) \\
  & P( D_{R} \mid T_{IA}, B_{JK} ) \\
  & P( T_{IA} \mid T_{I}, X_{IA}, e_{IA} ) \\
  & P( T_{I} \mid X_{I}, e_{I} ) \\
  & P( B_{JK} \mid B_{J}, Z_{JK}, e_{JK} ) \\
  & P( B_{J} \mid Z_{J}, e_{J} ) \\ \\
  & P( e_{I} ) P( e_{IA} ) P( e_{J} ) P( e_{JK} ) \\ \\ \\
\end{aligned}
\]

}

\subcaption{\label{fig-ITCJ_prob}Probabilistic model}

\end{minipage}%

\caption{\label{fig-ITCJ}ITCJ analysis. SCM (left) and probabilistic
(right) representations for DAG in Figure~\ref{fig-ITCJ_dag}.}

\end{figure}%

Critically, the approach allows tailoring this general structure to a
specific CJ context, enabling development of parsimonious models that
match the assumed data-generating process without imposing unnecessary
constraints. For example, \citet{Rivera_et_al_2025} modeled a CJ
assessment designed to evaluate the impact of different teaching methods
on students' writing ability. In this case, the observed outcome was
binary, so the model assumed \(O_{R}\) followed a Bernoulli
distribution. The discriminal difference \((D_{R})\) was determined by
the texts' discriminal processes \((T_{IA})\) and the judgment biases
\((B_{JK})\). Student-level variables \(X_{I}\), such as teaching
method, were included, whereas text-level variables \(X_{IA}\) (e.g.,
text length) were not gathered. Similarly, judge-level variables
\(Z_{J}\), like judgment expertise, were incorporated, while
judgment-level variables \(Z_{JK}\) (e.g., number of judgments per
judge) were absent. Finally, the probabilistic assumptions for the
idiosyncratic errors (\(e_{I}\), \(e_{IA}\), \(e_{J}\), \(e_{JK}\))
resolved indeterminacies in \emph{location}, \emph{orientation}, and
\emph{scale} for the variables \(T_{I}\), \(T_{IA}\), \(B_{J}\),
\(B_{JK}\), as required in latent variable models
\citep{Depaoli_2021, deAyala_2009}.

Lastly, researcher can derive one or more \emph{bespoke} statistical
models tailored to the CJ system of interest, as demonstrated in
\citet{Rivera_et_al_2025}. At this stage, the ITCJ analysis differs
fundamentally from the CBTL approach in how it handles parameter
estimation and inference. Rather than fitting multiple separate models,
the approach simultaneously estimates all parameters within a single
coherent framework using \emph{Bayesian inference}. This joint
estimation accounts for uncertainty at all levels and enables direct
inference about quantities of interest without relying on post-hoc
procedures \citep{McElreath_2020}.

\section{Methods}\label{sec-methods}

\subsection{Step 1, from Theory to Design: Data-generating
assumptions}\label{sec-methods_step1}

\subsection{Step 2, from Design to Data: Data
simulation}\label{sec-methods_step2}

\subsection{Step 5, from Estimator and Sample to Estimate(s): The
analysis aproaches}\label{sec-methods_step5}

\subsubsection{The CBTL analysis}\label{sec-methods_step5_1}

\subsubsection{The ITCJ analysis}\label{sec-methods_step5_2}

\paragraph{Model 1}\label{sec-sec-methods_step5_2_itcj_1}

\paragraph{Model 2}\label{sec-sec-methods_step5_2_itcj_2}

\paragraph{Model 3}\label{sec-sec-methods_step5_2_itcj_3}

\paragraph{Model 4}\label{sec-sec-methods_step5_2_itcj_4}

\paragraph{Model 5}\label{sec-sec-methods_step5_2_itcj_5}

\paragraph{Model 6}\label{sec-sec-methods_step5_2_itcj_6}

\subsection{Step 6, from Estimate(s) to Diagnostics and Posterior
predictives: The evaluation criteria}\label{sec-methods_step6}

\section{Results}\label{sec-results}

\subsection{Data description}\label{sec-results_data}

\subsection{Data modeling}\label{sec-results_modeling}

\subsubsection{The CBTL analysis}\label{sec-results_modeling_1}

\subsubsection{The ITCJ analysis}\label{sec-results_modeling_2}

\paragraph{Model 1}\label{sec-results_modeling_2_1}

\paragraph{Model 2}\label{sec-results_modeling_2_2}

\paragraph{Model 3}\label{sec-results_modeling_2_3}

\paragraph{Model 4}\label{sec-results_modeling_2_4}

\paragraph{Model 5}\label{sec-results_modeling_2_5}

\paragraph{Model 6}\label{sec-results_modeling_2_6}

\paragraph{Model comparison}\label{sec-results_modeling_2_7}

\section{Discussion}\label{sec-discussion}

\subsection{Future research directions}\label{sec-discussion_RD}

\subsection{Study limitations}\label{sec-discussion_limitations}

\section{Conclusion}\label{sec-conclusion}

\newpage{}

\section*{Declarations}\label{declarations}
\addcontentsline{toc}{section}{Declarations}

\textbf{Funding:} The Research Fund (BOF) of the University of Antwerp
funded this project.

\textbf{Financial interests:} The authors declare no relevant financial
interests.

\textbf{Non-financial interests:} The authors declare no relevant
non-financial interests.

\textbf{Ethics approval:} The University of Antwerp Research Ethics
Committee confirmed that this study does not require ethical approval.

\textbf{Consent to participate:} Not applicable

\textbf{Consent for publication:} All authors have read and approved the
final version of the manuscript for publication.

\textbf{Data, materials and code availability:} A previous version of
this manuscript, along with the associated data, materials and code (see
the section titled \texttt{CODE\ LINK}), has been made publicly
available at: \url{https://jriveraespejo.github.io/paper3_manuscript/}.

\textbf{Licence:} All the code that is original to this study and not
attributed to any other authors is copyrighted by
\href{https://orcid.org/0000-0002-3088-2783}{Jose Manuel Rivera Espejo}
and released under the new
\href{https://opensource.org/license/BSD-3-Clause}{BSD-3-Clause}
license.

\textbf{AI-assisted technologies in the writing process:} The authors
used various AI-based language tools to refine phrasing, optimize
wording, and enhance clarity and coherence throughout the manuscript.
They take full responsibility for the final content of the publication.

\textbf{CRediT authorship contribution statement:}
\emph{Conceptualization:} J.M.R.E, T.vD., S.DM., and S.G.;
\emph{Methodology:} J.M.R.E, T.vD., and S.DM.; \emph{Software:}
J.M.R.E.; \emph{Validation:} J.M.R.E.; \emph{Formal Analysis:} J.M.R.E.;
\emph{Investigation:} J.M.R.E; \emph{Resources:} T.vD. and S.DM.;
\emph{Data curation:} J.M.R.E.; \emph{Writing - original draft:}
J.M.R.E.; \emph{Writing - review and editing:} J.M.R.E., T.vD., S.DM.,
and S.G.; \emph{Visualization:} J.M.R.E.; \emph{Supervision:} S.G. and
S.DM.; \emph{Project administration:} S.G. and S.DM.; \emph{Funding
acquisition:} S.G. and S.DM.

\newpage{}

\section{Appendix}\label{sec-appendix}

\subsection{Appendix A: Stationarity, converge and
mixing}\label{sec-appendixA}

\subsection{Appendix B: Misfit observations}\label{sec-appendixB}

\subsection{Appendix C: Sample size calculations}\label{sec-appendixC}

\newpage{}

\section*{References}\label{references}
\addcontentsline{toc}{section}{References}

\renewcommand{\bibsection}{}
\bibliography{references.bib}





\end{document}
