% Options for packages loaded elsewhere
% Options for packages loaded elsewhere
\PassOptionsToPackage{unicode}{hyperref}
\PassOptionsToPackage{hyphens}{url}
\PassOptionsToPackage{dvipsnames,svgnames,x11names}{xcolor}
%
\documentclass[
  authoryear,
  review,
  1p]{elsarticle}
\usepackage{xcolor}
\usepackage{amsmath,amssymb}
\setcounter{secnumdepth}{5}
\usepackage{iftex}
\ifPDFTeX
  \usepackage[T1]{fontenc}
  \usepackage[utf8]{inputenc}
  \usepackage{textcomp} % provide euro and other symbols
\else % if luatex or xetex
  \usepackage{unicode-math} % this also loads fontspec
  \defaultfontfeatures{Scale=MatchLowercase}
  \defaultfontfeatures[\rmfamily]{Ligatures=TeX,Scale=1}
\fi
\usepackage{lmodern}
\ifPDFTeX\else
  % xetex/luatex font selection
\fi
% Use upquote if available, for straight quotes in verbatim environments
\IfFileExists{upquote.sty}{\usepackage{upquote}}{}
\IfFileExists{microtype.sty}{% use microtype if available
  \usepackage[]{microtype}
  \UseMicrotypeSet[protrusion]{basicmath} % disable protrusion for tt fonts
}{}
\makeatletter
\@ifundefined{KOMAClassName}{% if non-KOMA class
  \IfFileExists{parskip.sty}{%
    \usepackage{parskip}
  }{% else
    \setlength{\parindent}{0pt}
    \setlength{\parskip}{6pt plus 2pt minus 1pt}}
}{% if KOMA class
  \KOMAoptions{parskip=half}}
\makeatother
% Make \paragraph and \subparagraph free-standing
\makeatletter
\ifx\paragraph\undefined\else
  \let\oldparagraph\paragraph
  \renewcommand{\paragraph}{
    \@ifstar
      \xxxParagraphStar
      \xxxParagraphNoStar
  }
  \newcommand{\xxxParagraphStar}[1]{\oldparagraph*{#1}\mbox{}}
  \newcommand{\xxxParagraphNoStar}[1]{\oldparagraph{#1}\mbox{}}
\fi
\ifx\subparagraph\undefined\else
  \let\oldsubparagraph\subparagraph
  \renewcommand{\subparagraph}{
    \@ifstar
      \xxxSubParagraphStar
      \xxxSubParagraphNoStar
  }
  \newcommand{\xxxSubParagraphStar}[1]{\oldsubparagraph*{#1}\mbox{}}
  \newcommand{\xxxSubParagraphNoStar}[1]{\oldsubparagraph{#1}\mbox{}}
\fi
\makeatother


\usepackage{longtable,booktabs,array}
\newcounter{none} % for unnumbered tables
\usepackage{calc} % for calculating minipage widths
% Correct order of tables after \paragraph or \subparagraph
\usepackage{etoolbox}
\makeatletter
\patchcmd\longtable{\par}{\if@noskipsec\mbox{}\fi\par}{}{}
\makeatother
% Allow footnotes in longtable head/foot
\IfFileExists{footnotehyper.sty}{\usepackage{footnotehyper}}{\usepackage{footnote}}
\makesavenoteenv{longtable}
\usepackage{graphicx}
\makeatletter
\newsavebox\pandoc@box
\newcommand*\pandocbounded[1]{% scales image to fit in text height/width
  \sbox\pandoc@box{#1}%
  \Gscale@div\@tempa{\textheight}{\dimexpr\ht\pandoc@box+\dp\pandoc@box\relax}%
  \Gscale@div\@tempb{\linewidth}{\wd\pandoc@box}%
  \ifdim\@tempb\p@<\@tempa\p@\let\@tempa\@tempb\fi% select the smaller of both
  \ifdim\@tempa\p@<\p@\scalebox{\@tempa}{\usebox\pandoc@box}%
  \else\usebox{\pandoc@box}%
  \fi%
}
% Set default figure placement to htbp
\def\fps@figure{htbp}
\makeatother





\setlength{\emergencystretch}{3em} % prevent overfull lines

\providecommand{\tightlist}{%
  \setlength{\itemsep}{0pt}\setlength{\parskip}{0pt}}



 
\usepackage[]{natbib}
\bibliographystyle{elsarticle-harv}


\makeatletter
\@ifpackageloaded{caption}{}{\usepackage{caption}}
\AtBeginDocument{%
\ifdefined\contentsname
  \renewcommand*\contentsname{Table of contents}
\else
  \newcommand\contentsname{Table of contents}
\fi
\ifdefined\listfigurename
  \renewcommand*\listfigurename{List of Figures}
\else
  \newcommand\listfigurename{List of Figures}
\fi
\ifdefined\listtablename
  \renewcommand*\listtablename{List of Tables}
\else
  \newcommand\listtablename{List of Tables}
\fi
\ifdefined\figurename
  \renewcommand*\figurename{Figure}
\else
  \newcommand\figurename{Figure}
\fi
\ifdefined\tablename
  \renewcommand*\tablename{Table}
\else
  \newcommand\tablename{Table}
\fi
}
\@ifpackageloaded{float}{}{\usepackage{float}}
\floatstyle{ruled}
\@ifundefined{c@chapter}{\newfloat{codelisting}{h}{lop}}{\newfloat{codelisting}{h}{lop}[chapter]}
\floatname{codelisting}{Listing}
\newcommand*\listoflistings{\listof{codelisting}{List of Listings}}
\makeatother
\makeatletter
\makeatother
\makeatletter
\@ifpackageloaded{caption}{}{\usepackage{caption}}
\@ifpackageloaded{subcaption}{}{\usepackage{subcaption}}
\makeatother
\journal{Psychological methods}
\usepackage{bookmark}
\IfFileExists{xurl.sty}{\usepackage{xurl}}{} % add URL line breaks if available
\urlstyle{same}
\hypersetup{
  pdftitle={Finding Thurstone: modeling comparative judgment data with R (and Stan)},
  pdfauthor={Jose Manuel Rivera Espejo; Tine van Daal; Sven De Maeyer; Steven Gillis},
  pdfkeywords={tutorial, causal inference, bayesian
inference, thurstonian model, comparative judgement, statistical
modeling},
  colorlinks=true,
  linkcolor={blue},
  filecolor={Maroon},
  citecolor={Blue},
  urlcolor={Blue},
  pdfcreator={LaTeX via pandoc}}


\setlength{\parindent}{6pt}
\begin{document}

\begin{frontmatter}
\title{Finding Thurstone: modeling comparative judgment data with
\texttt{R} (and \texttt{Stan})}
\author[1]{Jose Manuel Rivera Espejo%
\corref{cor1}%
}
 \ead{JoseManuel.RiveraEspejo@uantwerpen.be} 
\author[1]{Tine Daal%
%
}
 \ead{tine.vandaal@uantwerpen.be} 
\author[1]{Sven Maeyer%
%
}
 \ead{sven.demaeyer@uantwerpen.be} 
\author[2]{Steven Gillis%
%
}
 \ead{steven.gillis@uantwerpen.be} 

\affiliation[1]{organization={University of Antwerp, Training and
education sciences},,postcodesep={}}
\affiliation[2]{organization={University of
Antwerp, Linguistics},,postcodesep={}}

\cortext[cor1]{Corresponding author}




        
\begin{abstract}
The classical BTL analysis has become the standard approach for
analyzing comparative judgment (CJ) data because it provides a simple
method for measuring traits and conducting related analysis. This
simplicity arises from two key features. First, the approach relies on
the Bradley-Terry-Luce (BTL) model to estimate latent traits. Second, it
uses of ad hoc procedures to conduct related analysis. However, recent
studies question whether the BTL model assumptions hold in contemporary
CJ applications and whether the ad hoc procedures effectively fulfill
their intended analytical goals.

To address these concerns, \citet{Rivera_et_al_2025} proposed an
approach that extends the general form of Thurstone's law of comparative
judgment. The approach enables the development of a model tailored to
the assumed data-generating process of the CJ system under study,
eliminating the need for simplifying assumptions. Moreover, by
integrating measurement and inference within a single analytical
framework, it also removes the dependence on ad hoc analytical
procedures. Despite these advantages, the approach still requires
empirical validation.

Thus, this study empirically validates the proposed
Information-Theoretical model for CJ, benchmarked against the classical
BTL analysis, and demonstrates its practical implementation. The
document includes a structured tutorial based on a simulated
speech-quality dataset, providing guidance on data simulation, prior
specification, model estimation, and interpretation using \texttt{Stan},
\texttt{R}, and the interface packages \texttt{cmdstan} and
\texttt{brms}. Ultimately, the study equips researchers with practical
tools to apply the model to more complex CJ studies.
\end{abstract}





\begin{keyword}
    tutorial \sep causal inference \sep bayesian
inference \sep thurstonian model \sep comparative judgement \sep 
    statistical modeling
\end{keyword}
\end{frontmatter}
    

\section{Introduction}\label{sec-introduction}

\emph{Comparative judgment} (CJ) has emerged as a valuable methodology
for measuring latent traits across diverse fields, including education
\citep{Kimbell_2012, Jones_et_al_2015, vanDaal_et_al_2016, Bartholomew_et_al_2018},
political sciences \citep{Zucco_et_al_2019}, linguistics
\citep{Boonen_et_al_2020}, and criminology \citep{Seymour_et_al_2025}.
In CJ studies, judges actively compare pairs of stimuli to determine
which stimulus exhibits more of the latent trait of interest
\citep{Thurstone_1927a, Thurstone_1927b}.

A specific data analysis workflow has become the standard approach for
analyzing CJ data \citep[see, e.g.,][]{Thwaites_et_al_2024}. In this
study, we refer to this workflow as the \emph{classical BTL analysis}.
Researchers favor this approach because it provides a simple method for
measuring traits and conducting related analyses
\citep{Andrich_1978, Pollitt_2012b}. This simplicity, in turn, arises
from two key features. First, the approach relies on the
Bradley-Terry-Luce (BTL) model \citep{Bradley_et_al_1952, Luce_1959} to
estimate latent traits. The model facilitates trait estimation by
imposing several simplifying assumptions about traits, judges, and
stimuli in CJ assessments \citep{Thurstone_1927b, Bramley_2008}. Second,
the approach uses ad hoc procedures to conduct related analysis,
including data summarization and statistical inference
\citep{Pollitt_2012b}.

Recent studies, however, question whether the assumptions of the BTL
model hold in contemporary CJ applications and whether the ad hoc
procedures achieve their intended analytical goals
\citep{Bramley_2008, Kelly_et_al_2022, Rivera_et_al_2025}. For instance,
Rivera and colleagues \citeyearpar{Rivera_et_al_2025} argue that while
the assumptions of equal dispersions and zero correlations between
stimuli simplify trait measurement, they may fail to represent complex
traits or heterogeneous stimuli adequately
\citep{Thurstone_1927a, Andrich_1978, vanDaal_et_al_2016, Lesterhuis_et_al_2018, Chambers_et_al_2022}.
As a result, such assumptions can compromise the reliability and
accuracy of trait estimates
\citep{Ackerman_1989, Zimmerman_1994, McElreath_2020, Wu_et_al_2022, Miller_2023, Hoyle_et_al_2023}.
Moreover, the same authors note that although ad hoc procedures simplify
data analyses, the use of untested methods can also undermine the
validity of statistical inferences derived from CJ data
\citep{McElreath_2020, Kline_et_al_2023, Hoyle_et_al_2023}.

To address these concerns, \citet{Rivera_et_al_2025} proposed an
approach that extends the general form of Thurstone's law of comparative
judgment \citep{Thurstone_1927a, Thurstone_1927b}. This approach
leverages causal and Bayesian inference methods to combine Thurstone's
core theoretical principles with key design features of CJ assessment.
By doing so, it enables the development of a model tailored to the
assumed data-generating process of the CJ system under study. This
tailoring effectively removes the need to rely on the simplifying
assumptions of the BTL model. Moreover, by integrating measurement and
inference within a single analytical framework, the approach also
eliminates the dependence on ad hoc analytical procedures. Ultimately,
this approach has the potential to produce reliable trait estimates and
accurate statistical inferences. However, its effectiveness still
required empirical validation.

\subsection{Research goals}\label{sec-introduction_goals}

Building on the introduction, this study pursues two closely related
research goals. The first goal is to \emph{empirically validate} the
proposed Information-Theoretical model for CJ (hereafter, ITCJ analysis)
by evaluating the accuracy and reliability of its trait estimates and
inference parameters, benchmarked against the classical BTL analysis
(hereafter, CBTL analysis). The second goal emerges as a practical
byproduct of this validation: to demonstrate how to implement the model
in practice. To this end, the document provides a structured tutorial
based on a simulated speech-quality dataset, offering guidance on data
simulation, prior specification, model estimation, and interpretation
using \texttt{Stan}\citep{Stan_2026a, Stan_2026b}, \texttt{R}
\citep{R_2015}, and the interface packages \texttt{cmdstan}
\citep{Gabry_et_al_2025} and \texttt{brms}
\citep{Burkner_2017, Burkner_2018}. By combining model validation and
practical instruction, the study can evaluate the methodological
performance of the ITCJ analysis and provide researchers with practical
tools to apply it to more complex CJ studies.

The remainder of this manuscript is organized into five sections.
Section~\ref{sec-theory} reviews the two analytical approaches commonly
applied to CJ data: the CBTL and ITCJ analyses.
Section~\ref{sec-methods} details the assumed data-generating process
for the simulated dataset, the simulation procedure, the practical
implementation of each analytical approach, and the evaluation criteria
aligned with the research goals. Section~\ref{sec-results} presents the
data description and modeling results. Section~\ref{sec-discussion}
interprets the findings, outlines future research directions, and
considers the study limitations. Finally, Section~\ref{sec-conclusion}
offers the concluding remarks.

\section{A tale of two analytical approaches}\label{sec-theory}

\subsection{The classical BTL analysis}\label{sec-theory_CBTL}

\subsection{The Information-Theoretical model for
CJ}\label{sec-theory_ITCJ}

\newcommand{\dsep}{\:\bot\:}
\newcommand{\ndsep}{\:\not\bot\:}

\section{Methods}\label{sec-methods}

\subsection{Step 1, from Theory to Design: Data-generating
assumptions}\label{sec-methods_step1}

\subsection{Step 2, from Design to Data: Data
simulation}\label{sec-methods_step2}

\subsection{Step 5, from Estimator and Sample to Estimate(s): The
analysis aproaches}\label{sec-methods_step5}

\subsubsection{The CBTL analysis}\label{sec-methods_step5_1}

\subsubsection{The ITCJ analysis}\label{sec-methods_step5_2}

\paragraph{Model 1}\label{sec-sec-methods_step5_2_itcj_1}

\paragraph{Model 2}\label{sec-sec-methods_step5_2_itcj_2}

\paragraph{Model 3}\label{sec-sec-methods_step5_2_itcj_3}

\paragraph{Model 4}\label{sec-sec-methods_step5_2_itcj_4}

\paragraph{Model 5}\label{sec-sec-methods_step5_2_itcj_5}

\paragraph{Model 6}\label{sec-sec-methods_step5_2_itcj_6}

\subsection{Step 6, from Estimate(s) to Diagnostics and Posterior
predictives: The evaluation criteria}\label{sec-methods_step6}

\section{Results}\label{sec-results}

\subsection{Data description}\label{sec-results_data}

\subsection{Data modeling}\label{sec-results_modeling}

\subsubsection{The CBTL analysis}\label{sec-results_modeling_1}

\subsubsection{The ITCJ analysis}\label{sec-results_modeling_2}

\paragraph{Model 1}\label{sec-results_modeling_2_1}

\paragraph{Model 2}\label{sec-results_modeling_2_2}

\paragraph{Model 3}\label{sec-results_modeling_2_3}

\paragraph{Model 4}\label{sec-results_modeling_2_4}

\paragraph{Model 5}\label{sec-results_modeling_2_5}

\paragraph{Model 6}\label{sec-results_modeling_2_6}

\paragraph{Model comparison}\label{sec-results_modeling_2_7}

\section{Discussion}\label{sec-discussion}

\subsection{Future research directions}\label{sec-discussion_RD}

\subsection{Study limitations}\label{sec-discussion_limitations}

\section{Conclusion}\label{sec-conclusion}

\newpage{}

\section*{Declarations}\label{declarations}
\addcontentsline{toc}{section}{Declarations}

\textbf{Funding:} The Research Fund (BOF) of the University of Antwerp
funded this project.

\textbf{Financial interests:} The authors declare no relevant financial
interests.

\textbf{Non-financial interests:} The authors declare no relevant
non-financial interests.

\textbf{Ethics approval:} The University of Antwerp Research Ethics
Committee confirmed that this study does not require ethical approval.

\textbf{Consent to participate:} Not applicable

\textbf{Consent for publication:} All authors have read and approved the
final version of the manuscript for publication.

\textbf{Data, materials and code availability:} A previous version of
this manuscript, along with the associated data, materials and code (see
the section titled \texttt{CODE\ LINK}), has been made publicly
available at: \url{https://jriveraespejo.github.io/paper3_manuscript/}.

\textbf{Licence:} All the code that is original to this study and not
attributed to any other authors is copyrighted by
\href{https://orcid.org/0000-0002-3088-2783}{Jose Manuel Rivera Espejo}
and released under the new
\href{https://opensource.org/license/BSD-3-Clause}{BSD-3-Clause}
license.

\textbf{AI-assisted technologies in the writing process:} The authors
used various AI-based language tools to refine phrasing, optimize
wording, and enhance clarity and coherence throughout the manuscript.
They take full responsibility for the final content of the publication.

\textbf{CRediT authorship contribution statement:}
\emph{Conceptualization:} J.M.R.E, T.vD., S.DM., and S.G.;
\emph{Methodology:} J.M.R.E, T.vD., and S.DM.; \emph{Software:}
J.M.R.E.; \emph{Validation:} J.M.R.E.; \emph{Formal Analysis:} J.M.R.E.;
\emph{Investigation:} J.M.R.E; \emph{Resources:} T.vD. and S.DM.;
\emph{Data curation:} J.M.R.E.; \emph{Writing - original draft:}
J.M.R.E.; \emph{Writing - review and editing:} J.M.R.E., T.vD., S.DM.,
and S.G.; \emph{Visualization:} J.M.R.E.; \emph{Supervision:} S.G. and
S.DM.; \emph{Project administration:} S.G. and S.DM.; \emph{Funding
acquisition:} S.G. and S.DM.

\newpage{}

\section{Appendix}\label{sec-appendix}

\subsection{Appendix A: Stationarity, converge and
mixing}\label{sec-appendixA}

\subsection{Appendix B: Misfit observations}\label{sec-appendixB}

\subsection{Appendix C: Sample size calculations}\label{sec-appendixC}

\newpage{}

\section*{References}\label{references}
\addcontentsline{toc}{section}{References}

\renewcommand{\bibsection}{}
\bibliography{references.bib}





\end{document}
