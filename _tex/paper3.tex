% Options for packages loaded elsewhere
% Options for packages loaded elsewhere
\PassOptionsToPackage{unicode}{hyperref}
\PassOptionsToPackage{hyphens}{url}
\PassOptionsToPackage{dvipsnames,svgnames,x11names}{xcolor}
%
\documentclass[
  authoryear,
  review,
  1p]{elsarticle}
\usepackage{xcolor}
\usepackage{amsmath,amssymb}
\setcounter{secnumdepth}{5}
\usepackage{iftex}
\ifPDFTeX
  \usepackage[T1]{fontenc}
  \usepackage[utf8]{inputenc}
  \usepackage{textcomp} % provide euro and other symbols
\else % if luatex or xetex
  \usepackage{unicode-math} % this also loads fontspec
  \defaultfontfeatures{Scale=MatchLowercase}
  \defaultfontfeatures[\rmfamily]{Ligatures=TeX,Scale=1}
\fi
\usepackage{lmodern}
\ifPDFTeX\else
  % xetex/luatex font selection
\fi
% Use upquote if available, for straight quotes in verbatim environments
\IfFileExists{upquote.sty}{\usepackage{upquote}}{}
\IfFileExists{microtype.sty}{% use microtype if available
  \usepackage[]{microtype}
  \UseMicrotypeSet[protrusion]{basicmath} % disable protrusion for tt fonts
}{}
\makeatletter
\@ifundefined{KOMAClassName}{% if non-KOMA class
  \IfFileExists{parskip.sty}{%
    \usepackage{parskip}
  }{% else
    \setlength{\parindent}{0pt}
    \setlength{\parskip}{6pt plus 2pt minus 1pt}}
}{% if KOMA class
  \KOMAoptions{parskip=half}}
\makeatother
% Make \paragraph and \subparagraph free-standing
\makeatletter
\ifx\paragraph\undefined\else
  \let\oldparagraph\paragraph
  \renewcommand{\paragraph}{
    \@ifstar
      \xxxParagraphStar
      \xxxParagraphNoStar
  }
  \newcommand{\xxxParagraphStar}[1]{\oldparagraph*{#1}\mbox{}}
  \newcommand{\xxxParagraphNoStar}[1]{\oldparagraph{#1}\mbox{}}
\fi
\ifx\subparagraph\undefined\else
  \let\oldsubparagraph\subparagraph
  \renewcommand{\subparagraph}{
    \@ifstar
      \xxxSubParagraphStar
      \xxxSubParagraphNoStar
  }
  \newcommand{\xxxSubParagraphStar}[1]{\oldsubparagraph*{#1}\mbox{}}
  \newcommand{\xxxSubParagraphNoStar}[1]{\oldsubparagraph{#1}\mbox{}}
\fi
\makeatother


\usepackage{longtable,booktabs,array}
\usepackage{calc} % for calculating minipage widths
% Correct order of tables after \paragraph or \subparagraph
\usepackage{etoolbox}
\makeatletter
\patchcmd\longtable{\par}{\if@noskipsec\mbox{}\fi\par}{}{}
\makeatother
% Allow footnotes in longtable head/foot
\IfFileExists{footnotehyper.sty}{\usepackage{footnotehyper}}{\usepackage{footnote}}
\makesavenoteenv{longtable}
\usepackage{graphicx}
\makeatletter
\newsavebox\pandoc@box
\newcommand*\pandocbounded[1]{% scales image to fit in text height/width
  \sbox\pandoc@box{#1}%
  \Gscale@div\@tempa{\textheight}{\dimexpr\ht\pandoc@box+\dp\pandoc@box\relax}%
  \Gscale@div\@tempb{\linewidth}{\wd\pandoc@box}%
  \ifdim\@tempb\p@<\@tempa\p@\let\@tempa\@tempb\fi% select the smaller of both
  \ifdim\@tempa\p@<\p@\scalebox{\@tempa}{\usebox\pandoc@box}%
  \else\usebox{\pandoc@box}%
  \fi%
}
% Set default figure placement to htbp
\def\fps@figure{htbp}
\makeatother





\setlength{\emergencystretch}{3em} % prevent overfull lines

\providecommand{\tightlist}{%
  \setlength{\itemsep}{0pt}\setlength{\parskip}{0pt}}



 
\usepackage[]{natbib}
\bibliographystyle{elsarticle-harv}


\makeatletter
\@ifpackageloaded{caption}{}{\usepackage{caption}}
\AtBeginDocument{%
\ifdefined\contentsname
  \renewcommand*\contentsname{Table of contents}
\else
  \newcommand\contentsname{Table of contents}
\fi
\ifdefined\listfigurename
  \renewcommand*\listfigurename{List of Figures}
\else
  \newcommand\listfigurename{List of Figures}
\fi
\ifdefined\listtablename
  \renewcommand*\listtablename{List of Tables}
\else
  \newcommand\listtablename{List of Tables}
\fi
\ifdefined\figurename
  \renewcommand*\figurename{Figure}
\else
  \newcommand\figurename{Figure}
\fi
\ifdefined\tablename
  \renewcommand*\tablename{Table}
\else
  \newcommand\tablename{Table}
\fi
}
\@ifpackageloaded{float}{}{\usepackage{float}}
\floatstyle{ruled}
\@ifundefined{c@chapter}{\newfloat{codelisting}{h}{lop}}{\newfloat{codelisting}{h}{lop}[chapter]}
\floatname{codelisting}{Listing}
\newcommand*\listoflistings{\listof{codelisting}{List of Listings}}
\makeatother
\makeatletter
\makeatother
\makeatletter
\@ifpackageloaded{caption}{}{\usepackage{caption}}
\@ifpackageloaded{subcaption}{}{\usepackage{subcaption}}
\makeatother
\journal{Psychological methods}
\usepackage{bookmark}
\IfFileExists{xurl.sty}{\usepackage{xurl}}{} % add URL line breaks if available
\urlstyle{same}
\hypersetup{
  pdftitle={Bayesian modeling of comparative judgment data with R and Stan: A tutorial for speech quality researchers},
  pdfauthor={Jose Manuel Rivera Espejo; Tine van Daal; Sven De Maeyer; Steven Gillis},
  pdfkeywords={tutorial, causal inference, bayesian
inference, thurstonian model, comparative judgement, statistical
modeling},
  colorlinks=true,
  linkcolor={blue},
  filecolor={Maroon},
  citecolor={Blue},
  urlcolor={Blue},
  pdfcreator={LaTeX via pandoc}}


\setlength{\parindent}{6pt}
\begin{document}

\begin{frontmatter}
\title{Bayesian modeling of comparative judgment data with \texttt{R}
and \texttt{Stan}: A tutorial for speech quality researchers}
\author[1]{Jose Manuel Rivera Espejo%
\corref{cor1}%
}
 \ead{JoseManuel.RiveraEspejo@uantwerpen.be} 
\author[1]{Tine Daal%
%
}
 \ead{tine.vandaal@uantwerpen.be} 
\author[1]{Sven Maeyer%
%
}
 \ead{sven.demaeyer@uantwerpen.be} 
\author[2]{Steven Gillis%
%
}
 \ead{steven.gillis@uantwerpen.be} 

\affiliation[1]{organization={University of Antwerp, Training and
education sciences},,postcodesep={}}
\affiliation[2]{organization={University of
Antwerp, Linguistics},,postcodesep={}}

\cortext[cor1]{Corresponding author}




        
\begin{abstract}
The Bradley-Terry-Luce (BTL) model is commonly used to analyze
comparative judgment (CJ) data because it provides a simple method for
measuring traits and conducting statistical inference. Its simplicity
stems from two key features: (1) a reliance on an extensive set of
simplifying assumptions about the traits, judges, and stimuli involved
in CJ assessments; and (2) the use of ad hoc procedures to handle
inferences, including hypothesis testing. However, recent literature
questions whether these assumptions hold in modern CJ applications and
whether the ad hoc procedures effectively fulfill their intended
analytical purpose.

To address these concerns, \citet{Rivera_et_al_2025} proposed an
approach that extends the general form of Thurstone's law of comparative
judgment. The approach enables the development of a model tailored to
the assumed data-generating process of the CJ system under study,
eliminating the need to rely on simplifying assumptions. Moreover, by
integrating measurement and inference within a single analytical
framework, the approach also removes the dependence on ad hoc
hypothesis-testing procedures.

This tutorial illustrates the application of the proposed approach to a
simulated dataset on speech quality. It offers detailed guidance on data
simulation, model specification, estimation, and interpretation using
the software \texttt{R} and \texttt{Stan}. While the tutorial assumes
familiarity with CJ theory and practice, latent variable models, and
causal inference, it does not require prior experience with Bayesian
inference methods or the associated software. Ultimately, by following
the outlined procedures, researchers can replicate this analysis and
adapt the approach for more complex CJ studies.
\end{abstract}





\begin{keyword}
    tutorial \sep causal inference \sep bayesian
inference \sep thurstonian model \sep comparative judgement \sep 
    statistical modeling
\end{keyword}
\end{frontmatter}
    

\newcommand{\dsep}{\:\bot\:}
\newcommand{\ndsep}{\:\not\bot\:}

\section{Introduction}\label{sec-introduction}

Comparative judgment (CJ) is an assessment method in which judges
evaluate a trait across different stimuli using pairwise comparisons
\citep{Thurstone_1927a, Thurstone_1927b}. Each comparison generates a
dichotomous outcome that indicates which stimulus is perceived to
exhibit a higher attribute level. For instance, judges might compare
pairs of short speech samples (the stimuli) to evaluate the relative
speech quality of children (the trait) \citep{Boonen_et_al_2020}.

The Bradley-Terry-Luce (BTL) model \citep{Bradley_et_al_1952, Luce_1959}
is then employed to analyze the CJ data, as it provides a simple method
for measuring traits and conducting statistical inference
\citep{Andrich_1978, Pollitt_2012b}. The method's simplicity stems from
two key features. First, it relies on an extensive set of simplifying
assumptions about the traits, judges, and stimuli involved in CJ
assessments \citep{Thurstone_1927b, Bramley_2008}. Second, it employs ad
hoc procedures to handle inferences, including hypothesis testing
\citep{Pollitt_2012b}.

However, recent studies question whether these assumptions hold in
modern CJ applications
\citep{Bramley_2008, Kelly_et_al_2022, Rivera_et_al_2025} and whether
the ad hoc procedures achieve their intended analytical purpose
\citep{Kelly_et_al_2022, Rivera_et_al_2025}. For instance,
\citet[pp.~2]{Rivera_et_al_2025} argues that while assuming equal
dispersions and zero correlation between stimuli simplifies the trait
measurement model, these assumptions may fail to capture the complexity
of some traits or account for heterogeneous stimuli
\citep{Thurstone_1927a, Andrich_1978, vanDaal_et_al_2016, Lesterhuis_et_al_2018, Chambers_et_al_2022}.
As a result, these assumptions can compromise the reliability and
accuracy of trait estimates
\citep{Ackerman_1989, Zimmerman_1994, McElreath_2020, Wu_et_al_2022, Miller_2023, Hoyle_et_al_2023}.
Moreover, the same authors note that although ad hoc procedures simplify
CJ data analysis, relying on untested methods can also undermine the
validity of statistical inferences drawn from the data
\citep{McElreath_2020, Kline_et_al_2023, Hoyle_et_al_2023}.

To address these concerns, \citet{Rivera_et_al_2025} proposed an
approach that extends the general form of Thurstone's law of comparative
judgment \citep{Thurstone_1927a, Thurstone_1927b} using causal and
Bayesian inference methods. The approach combines Thurstone's core
theoretical principles with key CJ assessment design features, enabling
the development of a model tailored to the assumed data-generating
process of the CJ system under study. This tailoring effectively
eliminates the need to rely on the simplifying assumptions of the BTL
model. Moreover, by integrating measurement and inference within a
single analytical framework, the approach also removes the dependence on
ad hoc procedures. Ultimately, the approach has the potential to yield
reliable trait estimates and accurate statistical inferences. However,
this promise still needs to be empirically tested.

Thus, this tutorial applies the proposed approach to a simulated dataset
on speech quality to evaluate whether the approach's promise holds in
practice. {Specifically, we are interested in the following research
questions:}. At the same time, it provides detailed guidance on data
simulation, model specification, estimation, and interpretation using
the software \texttt{R} and \texttt{Stan}. Notably, while the tutorial
assumes familiarity with CJ theory and practice, latent variable models,
and causal inference, it does not require prior experience with Bayesian
inference methods or the associated software. Ultimately, by following
the procedures here outlined, researchers can replicate the analysis and
adapt the approach to more complex CJ studies.

The remainder of this manuscript is organized into four sections.
Section~\ref{sec-methods} describes the model specification, the dataset
simulation, inference procedure, and evaluation metrics relevant to the
research questions. Section~\ref{sec-results} summarizes the analysis,
including parameter estimates, credible intervals, and comparisons with
the standard BTL model. Next, Section~\ref{sec-discussion} reviews the
findings, outlines future research directions, and discusses the
limitation of the study. Finally, Section~\ref{sec-conclusion} provides
the concluding remarks.

\section{Methods}\label{sec-methods}

\subsection{Model specification}\label{sec-model}

\subsection{Dataset simulation}\label{sec-simulation}

\subsection{Inference procedure}\label{sec-inference}

\subsection{Evaluation metrics}\label{sec-evaluation}

\section{Results}\label{sec-results}

\section{Discussion}\label{sec-discussion}

\subsection{Future research directions}\label{sec-discussion_RD}

\subsection{Study limitations}\label{sec-discussion_limitations}

\section{Conclusion}\label{sec-conclusion}

\newpage{}

\section*{Declarations}\label{declarations}
\addcontentsline{toc}{section}{Declarations}

\textbf{Funding:} The Research Fund (BOF) of the University of Antwerp
funded this project.

\textbf{Financial interests:} The authors declare no relevant financial
interests.

\textbf{Non-financial interests:} The authors declare no relevant
non-financial interests.

\textbf{Ethics approval:} The University of Antwerp Research Ethics
Committee confirmed that this study does not require ethical approval.

\textbf{Consent to participate:} Not applicable

\textbf{Consent for publication:} All authors have read and approved the
final version of the manuscript for publication.

\textbf{Data availability:} This study did not use any data.

\textbf{Materials and code availability:} A previous version of this
manuscript, along with the associated materials and code (see the
section titled \texttt{CODE\ LINK}), has been made publicly available
at: \url{https://jriveraespejo.github.io/paper3_manuscript/}.

\textbf{AI-assisted technologies in the writing process:} The authors
used various AI-based language tools to refine phrasing, optimize
wording, and enhance clarity and coherence throughout the manuscript.
They take full responsibility for the final content of the publication.

\textbf{CRediT authorship contribution statement:}
\emph{Conceptualization:} J.M.R.E, T.vD., S.DM., and S.G.;
\emph{Methodology:} J.M.R.E, T.vD., and S.DM.; \emph{Software:}
J.M.R.E.; \emph{Validation:} J.M.R.E.; \emph{Formal Analysis:} J.M.R.E.;
\emph{Investigation:} J.M.R.E; \emph{Resources:} T.vD. and S.DM.;
\emph{Data curation:} J.M.R.E.; \emph{Writing - original draft:}
J.M.R.E.; \emph{Writing - review and editing:} J.M.R.E., T.vD., S.DM.,
and S.G.; \emph{Visualization:} J.M.R.E.; \emph{Supervision:} S.G. and
S.DM.; \emph{Project administration:} S.G. and S.DM.; \emph{Funding
acquisition:} S.G. and S.DM.

\newpage{}

\newpage{}

\section*{References}\label{references}
\addcontentsline{toc}{section}{References}

\renewcommand{\bibsection}{}
\bibliography{references.bib}





\end{document}
