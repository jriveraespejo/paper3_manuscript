% Options for packages loaded elsewhere
% Options for packages loaded elsewhere
\PassOptionsToPackage{unicode}{hyperref}
\PassOptionsToPackage{hyphens}{url}
\PassOptionsToPackage{dvipsnames,svgnames,x11names}{xcolor}
%
\documentclass[
  authoryear,
  review,
  1p]{elsarticle}
\usepackage{xcolor}
\usepackage{amsmath,amssymb}
\setcounter{secnumdepth}{5}
\usepackage{iftex}
\ifPDFTeX
  \usepackage[T1]{fontenc}
  \usepackage[utf8]{inputenc}
  \usepackage{textcomp} % provide euro and other symbols
\else % if luatex or xetex
  \usepackage{unicode-math} % this also loads fontspec
  \defaultfontfeatures{Scale=MatchLowercase}
  \defaultfontfeatures[\rmfamily]{Ligatures=TeX,Scale=1}
\fi
\usepackage{lmodern}
\ifPDFTeX\else
  % xetex/luatex font selection
\fi
% Use upquote if available, for straight quotes in verbatim environments
\IfFileExists{upquote.sty}{\usepackage{upquote}}{}
\IfFileExists{microtype.sty}{% use microtype if available
  \usepackage[]{microtype}
  \UseMicrotypeSet[protrusion]{basicmath} % disable protrusion for tt fonts
}{}
\makeatletter
\@ifundefined{KOMAClassName}{% if non-KOMA class
  \IfFileExists{parskip.sty}{%
    \usepackage{parskip}
  }{% else
    \setlength{\parindent}{0pt}
    \setlength{\parskip}{6pt plus 2pt minus 1pt}}
}{% if KOMA class
  \KOMAoptions{parskip=half}}
\makeatother
% Make \paragraph and \subparagraph free-standing
\makeatletter
\ifx\paragraph\undefined\else
  \let\oldparagraph\paragraph
  \renewcommand{\paragraph}{
    \@ifstar
      \xxxParagraphStar
      \xxxParagraphNoStar
  }
  \newcommand{\xxxParagraphStar}[1]{\oldparagraph*{#1}\mbox{}}
  \newcommand{\xxxParagraphNoStar}[1]{\oldparagraph{#1}\mbox{}}
\fi
\ifx\subparagraph\undefined\else
  \let\oldsubparagraph\subparagraph
  \renewcommand{\subparagraph}{
    \@ifstar
      \xxxSubParagraphStar
      \xxxSubParagraphNoStar
  }
  \newcommand{\xxxSubParagraphStar}[1]{\oldsubparagraph*{#1}\mbox{}}
  \newcommand{\xxxSubParagraphNoStar}[1]{\oldsubparagraph{#1}\mbox{}}
\fi
\makeatother


\usepackage{longtable,booktabs,array}
\usepackage{calc} % for calculating minipage widths
% Correct order of tables after \paragraph or \subparagraph
\usepackage{etoolbox}
\makeatletter
\patchcmd\longtable{\par}{\if@noskipsec\mbox{}\fi\par}{}{}
\makeatother
% Allow footnotes in longtable head/foot
\IfFileExists{footnotehyper.sty}{\usepackage{footnotehyper}}{\usepackage{footnote}}
\makesavenoteenv{longtable}
\usepackage{graphicx}
\makeatletter
\newsavebox\pandoc@box
\newcommand*\pandocbounded[1]{% scales image to fit in text height/width
  \sbox\pandoc@box{#1}%
  \Gscale@div\@tempa{\textheight}{\dimexpr\ht\pandoc@box+\dp\pandoc@box\relax}%
  \Gscale@div\@tempb{\linewidth}{\wd\pandoc@box}%
  \ifdim\@tempb\p@<\@tempa\p@\let\@tempa\@tempb\fi% select the smaller of both
  \ifdim\@tempa\p@<\p@\scalebox{\@tempa}{\usebox\pandoc@box}%
  \else\usebox{\pandoc@box}%
  \fi%
}
% Set default figure placement to htbp
\def\fps@figure{htbp}
\makeatother





\setlength{\emergencystretch}{3em} % prevent overfull lines

\providecommand{\tightlist}{%
  \setlength{\itemsep}{0pt}\setlength{\parskip}{0pt}}



 
\usepackage[]{natbib}
\bibliographystyle{elsarticle-harv}


\makeatletter
\@ifpackageloaded{caption}{}{\usepackage{caption}}
\AtBeginDocument{%
\ifdefined\contentsname
  \renewcommand*\contentsname{Table of contents}
\else
  \newcommand\contentsname{Table of contents}
\fi
\ifdefined\listfigurename
  \renewcommand*\listfigurename{List of Figures}
\else
  \newcommand\listfigurename{List of Figures}
\fi
\ifdefined\listtablename
  \renewcommand*\listtablename{List of Tables}
\else
  \newcommand\listtablename{List of Tables}
\fi
\ifdefined\figurename
  \renewcommand*\figurename{Figure}
\else
  \newcommand\figurename{Figure}
\fi
\ifdefined\tablename
  \renewcommand*\tablename{Table}
\else
  \newcommand\tablename{Table}
\fi
}
\@ifpackageloaded{float}{}{\usepackage{float}}
\floatstyle{ruled}
\@ifundefined{c@chapter}{\newfloat{codelisting}{h}{lop}}{\newfloat{codelisting}{h}{lop}[chapter]}
\floatname{codelisting}{Listing}
\newcommand*\listoflistings{\listof{codelisting}{List of Listings}}
\makeatother
\makeatletter
\makeatother
\makeatletter
\@ifpackageloaded{caption}{}{\usepackage{caption}}
\@ifpackageloaded{subcaption}{}{\usepackage{subcaption}}
\makeatother
\journal{Psychological methods}
\usepackage{bookmark}
\IfFileExists{xurl.sty}{\usepackage{xurl}}{} % add URL line breaks if available
\urlstyle{same}
\hypersetup{
  pdftitle={Advanced modeling of comparative judgment data: Applications to speech quality},
  pdfauthor={Jose Manuel Rivera Espejo; Tine van Daal; Sven De Maeyer; Steven Gillis},
  pdfkeywords={causal inference, directed acyclic graphs, structural
causal models, bayesian statistical methods, thurstonian
model, comparative judgement, probability, statistical modeling},
  colorlinks=true,
  linkcolor={blue},
  filecolor={Maroon},
  citecolor={Blue},
  urlcolor={Blue},
  pdfcreator={LaTeX via pandoc}}


\setlength{\parindent}{6pt}
\begin{document}

\begin{frontmatter}
\title{Advanced modeling of comparative judgment data: Applications to
speech quality}
\author[1]{Jose Manuel Rivera Espejo%
\corref{cor1}%
}
 \ead{JoseManuel.RiveraEspejo@uantwerpen.be} 
\author[1]{Tine Daal%
%
}
 \ead{tine.vandaal@uantwerpen.be} 
\author[1]{Sven Maeyer%
%
}
 \ead{sven.demaeyer@uantwerpen.be} 
\author[2]{Steven Gillis%
%
}
 \ead{steven.gillis@uantwerpen.be} 

\affiliation[1]{organization={University of Antwerp, Training and
education sciences},,postcodesep={}}
\affiliation[2]{organization={University of
Antwerp, Linguistics},,postcodesep={}}

\cortext[cor1]{Corresponding author}




        
\begin{abstract}
Comparative judgment (CJ) data is often analyzed using the
Bradley-Terry-Luce (BTL) model, which offers a straightforward method
for measuring traits and conducting statistical inference. However,
despite its usefulness, research has indicated that several core
assumptions of the BTL model are rarely satisfied in modern CJ
applications. As a result, the model may struggle to capture the
complexity of some traits or stimuli, compromising the reliability and
accuracy of trait estimates. Additionally, its requirement to separate
trait measurement from hypothesis testing can further undermine the
accuracy of statistical inferences drawn from such data.

To address these limitations, \citet{Rivera_et_al_2025} proposed an
approach that extends Thurstone's general form using causal and Bayesian
inference methods. This approach allows for the construction of a model
specifically tailored to the assumptions of the CJ data under study.
Furthermore, it integrates measurement and hypothesis testing within a
single analytical framework, facilitating precise and accurate
inferences from the data.

This tutorial illustrates the application of the proposed approach to a
simulated dataset on speech quality. It provides detailed guidance on
data simulation, model specification, estimation, and interpretation
using the software \texttt{R} and \texttt{Stan}. The tutorial assumes
that researchers are familiar with causal and Bayesian inference
methods, as well as latent variable models, but may not have prior
experience with CJ data or the software. By following the procedures
outlined here, researchers can reproduce the analysis and adapt the
approach to other CJ studies.
\end{abstract}





\begin{keyword}
    causal inference \sep directed acyclic graphs \sep structural causal
models \sep bayesian statistical methods \sep thurstonian
model \sep comparative judgement \sep probability \sep 
    statistical modeling
\end{keyword}
\end{frontmatter}
    

\newcommand{\dsep}{\:\bot\:}
\newcommand{\ndsep}{\:\not\bot\:}

\section{Introduction}\label{sec-introduction}

\section{Data \& Methods}\label{sec-data-methods}

\section{Discussion}\label{sec-discussion}

\subsection{Future research directions}\label{sec-discussion_RD}

\subsection{Study limitations}\label{sec-discussion_limitations}

\section{Conclusion}\label{sec-conclusion}

\newpage{}

\section*{Declarations}\label{declarations}
\addcontentsline{toc}{section}{Declarations}

\textbf{Funding:} The Research Fund (BOF) of the University of Antwerp
funded this project.

\textbf{Financial interests:} The authors declare no relevant financial
interests.

\textbf{Non-financial interests:} The authors declare no relevant
non-financial interests.

\textbf{Ethics approval:} The University of Antwerp Research Ethics
Committee confirmed that this study does not require ethical approval.

\textbf{Consent to participate:} Not applicable

\textbf{Consent for publication:} All authors have read and approved the
final version of the manuscript for publication.

\textbf{Data availability:} This study did not use any data.

\textbf{Materials and code availability:} The \texttt{CODE\ LINK}
section at the top of the digital document located at:
\url{https://jriveraespejo.github.io/paper3_manuscript/} provides access
to all materials and code.

\textbf{AI-assisted technologies in the writing process:} The authors
used various AI-based language tools to refine phrasing, optimize
wording, and enhance clarity and coherence throughout the manuscript.
They take full responsibility for the final content of the publication.

\textbf{CRediT authorship contribution statement:}
\emph{Conceptualization:} J.M.R.E, T.vD., S.DM., and S.G.;
\emph{Methodology:} J.M.R.E, T.vD., and S.DM.; \emph{Software:}
J.M.R.E.; \emph{Validation:} J.M.R.E.; \emph{Formal Analysis:} J.M.R.E.;
\emph{Investigation:} J.M.R.E; \emph{Resources:} T.vD. and S.DM.;
\emph{Data curation:} J.M.R.E., S.G.; \emph{Writing - original draft:}
J.M.R.E.; \emph{Writing - review and editing:} T.vD., S.DM., and S.G.;
\emph{Visualization:} J.M.R.E.; \emph{Supervision:} S.G. and S.DM.;
\emph{Project administration:} S.G. and S.DM.; \emph{Funding
acquisition:} S.G. and S.DM.

\newpage{}

\newpage{}

\section*{References}\label{references}
\addcontentsline{toc}{section}{References}

\renewcommand{\bibsection}{}
\bibliography{references.bib}





\end{document}
